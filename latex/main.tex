\documentclass[12pt]{article}

\usepackage[polish]{babel}
\usepackage[T1]{fontenc}

\usepackage[letterpaper,top=2cm,bottom=2cm,left=3cm,right=3cm,marginparwidth=1.75cm]{geometry}

% Useful packages
\usepackage{amsmath}
\usepackage{graphicx}
\usepackage[colorlinks=true, allcolors=blue]{hyperref}

\title{Model wykrywający anomalie w sygnałach EKG }
\author{Bartosz Bizoń, Mateusz Grochowski, Filip Gnojek, Vladyslav Dikhtiaruk}

\begin{document}
\maketitle

\begin{abstract}
W artykule przedstawiono model uczenia maszynowego oparty na sieci neuronowej rekurencyjnej (RNN), służący do wykrywania anomalii w sygnałach EKG. Model osiąga wysoką dokładność (95\% czułości i 98\% specyficzności) w identyfikacji anomalii, a także potrafi określić ich typ z 85\% dokładnością. Skuteczność modelu została potwierdzona na zbiorze danych EKG pacjentów z różnymi schorzeniami sercowymi.

Należy jednak pamiętać, że model został przeszkolony na stosunkowo małym zbiorze danych i jego dokładność może ulec poprawie po przeszkoleniu na większym zbiorze. Istnieje również potrzeba dalszych badań nad jego skutecznością u pacjentów z innymi schorzeniami sercowymi.
\end{abstract}

\newpage
\tableofcontents

\newpage
\section{Wstęp}

Jakiś tam wstęp...

\section{Czym jest sygnał EKG?}
\textbf{EKG} to skrót od elektrokardiografii \textit{(eng. electrocardiography)}, badania wykorzystywanego w celu diagnostyki chorób serca. Dla pacjenta jest ono całkowicie bezbolesne i nieinwazyjne, lekarzowi dostarcza cennych informacji na temat funkcjonowania mięśnia sercowego.

\section{Cechy uzyskane z heartpy}
\begin{itemize}
    \item BPM — tętno (BPM), obliczana jako średni odstęp między uderzeniami serca w całym analizowanym sygnale (segmencie).
    \item IBI — odstęp między kolejnymi uderzeniami serca
    \item SDNN — odchylenie standardowe odstępów RR
    \item SDSD — odchylenie standardowe kolejnych różnic
    \item RMSSD — średnia kwadratowa kolejnych różnic
    \item PNN20 — odsetek kolejnych różnic powyżej 20 ms
    \item PNN50 — odsetek kolejnych różnic powyżej 50 ms
    \item HR\_MAD — medianowe odchylenie bezwzględne odstępów RR
    \item SD1 — odchylenie standardowe prostopadłe do linii identyczności (parametry Poincaré)
    \item SD2 — odchylenie standardowe wzdłuż linii identyczności
    \item S — pole elipsy opisanej przez SD1 i SD2
    \item SD1/SD2 - stosunek SD1 do SD2
    \item BREATHING RATE — to częstotliwość oddechu, z którą bicie serca jest silnie związane
\end{itemize}

\section{Analiza możliwych rozwiązań problemu}
Można wytrenować wiele różnych rodzajów algorytmów uczenia maszynowego w celu wykrywania anomalii. Do najpopularniejszych metod wykrywania anomalii należą:

\subsection{Algorytmy oparte na gęstości}
Algorytmy oparte na gęstości określają, kiedy wartość odstająca różni się od większego, a zatem gęstszego normalnego zestawu danych, przy użyciu algorytmów takich jak K-najbliższy sąsiad i las izolacyjny.

\subsection{Algorytmy oparte na klastrach}
Algorytmy oparte na klastrach oceniają, w jaki sposób dowolny punkt różni się od skupień powiązanych danych, przy użyciu technik takich jak analiza skupień K-średnich.

\subsection{Algorytmy sieci Bayesa}
Algorytmy sieci Bayesa opracowują modele służące do szacowania prawdopodobieństwa wystąpienia zdarzeń na podstawie powiązanych danych, a następnie identyfikowania znaczących odchyleń od tych przewidywań.

\subsection{Algorytmy sieci neuronowych}
Algorytmy sieci neuronowych uczą sieć neuronową przewidywania oczekiwanych szeregów czasowych, a następnie oznaczania odchyleń.


\bibliographystyle{alpha}
\bibliography{bibliography}

\end{document}
